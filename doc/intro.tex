In questo progetto affrontiamo il problema della stima dell'età di un essere umano a partire da immagini del volto attraverso l'utilizzo di reti neurali profonde. 
In particolare, lo scopo che ci prefiggiamo è quello di addestrare una rete neurale profonda sul dataset \emph{VGG-Face2 Mivia Age}, anche detto \emph{VMAGE} \cite{miviaage}, che garantisca le migliori performance possibili sul test set associato a questo dataset, rispettando i seguenti requisiti.

\begin{itemize}
	\item Non richieda, per quanto possibile, risorse computazionali eccessive, dal momento che per l'addestramento sarà utilizzata la piattaforma Google Colab\footnote{\url{https://colab.research.google.com/}}, che offre potenza computazionale limitata e permette l'accesso alle sue risorse per un massimo di 12 ore consecutive, per poi sottoporre l'utente ad un ban temporaneo.
	\item \label{sec:intro.intention}Sia robusta rispetto a corruzioni che possono verificarsi su immagini acquisite con un'intenzione simile a quella con la quale sono state acquisite le immagini presenti nel dataset.
Essendo le immagini nel dataset immagini di celebrità recuperate su Google Immagini \cite{vggface2dataset}, si assumerà come intenzione quella di un fotografo che vuole realizzare uno scatto di un soggetto più o meno collaborativo.
	\item Sia implementata con il framework Keras, in quanto ci è più familiare, e questo ci permette di concentrarci sul modello senza dover spendere troppo tempo ad apprendere nuove tecniche implementative.
\end{itemize}

Il codice per l’addestramento e il test della rete realizzata nell’ambito di questo progetto è consultabile sulla nostra repository GitHub\footnote{\url{https://github.com/MicheleOliva/Contest-Vision-2020}}.