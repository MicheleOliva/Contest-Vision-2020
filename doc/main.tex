\documentclass[a4paper, 11pt]{article}
\usepackage{layaureo}
%Pacchetti per l'italiano
\usepackage{lmodern}
\usepackage[T1]{fontenc}
\usepackage[utf8]{inputenc}
\usepackage[italian]{babel}
\usepackage{csquotes}
% Pacchetto per eliminare le giunture
\usepackage{microtype}
% pacchetto teoremi ecc.
\usepackage{amsthm}
% Pacchetto matematica
\usepackage{amsmath}
\usepackage{amssymb}
% Pacchetto per gestire meglio i float (H e box)
\usepackage{float}
% pacchetto per aggiungere figure grafiche
\usepackage{graphicx}
% Pacchetto per aumentare il limite di float
\usepackage{morefloats}
% pacchetto per modificare le caption
\usepackage{caption}
% Pacchetto gestione dei hyperlinks
\usepackage{hyperref}
%liste in più
\usepackage{scrextend}
%liste inline
\usepackage[inline]{enumitem}
%autor
\usepackage{authblk}
% TO DO
\usepackage{todonotes}

\graphicspath{{Images/}}
\usepackage{wrapfig}
\usepackage{subcaption}
\usepackage{calc}

% Bibliografia
\usepackage[
backend=biber,
style=alphabetic,
sorting=ynt
]{biblatex}

%authblk in Italiano
\renewcommand\Authand{ e }
\renewcommand\Authands{ e }

%colora i link
\hypersetup{
    colorlinks=true,
    linkcolor=black,
    urlcolor=blue,
}

%Bibliografia
\addbibresource{biblio.bib}

% Titolo
\title{Contest di Visione Artificiale: Gruppo 2}
\author{Luca Boccia}
\author{Francesco Chiarello}
\author{Michele Oliva}
\affil{\texttt{\{\href{mailto:l.boccia12@studenti.unisa.it}{l.boccia12}, \href{mailto:f.chiarello1@studenti.unisa.it}{f.chiarello1}, \href{mailto:m.oliva26@studenti.unisa.it}{m.oliva26}\}@studenti.unisa.it}}
\affil{Università degli Studi di Salerno}
\date{Gennaio 2021}


\begin{document}

\maketitle
\tableofcontents
\listoffigures

\section{Introduzione}
In questo progetto affrontiamo il problema della stima dell'età di un essere umano a partire da immagini del volto attraverso l'utilizzo di reti neurali profonde. 
In particolare, lo scopo che ci prefiggiamo è quello di addestrare una rete neurale profonda sul dataset VGGFace2 \cite{vggface2dataset}\todo{Il dataset si chiama VMAGE, e spiegare che è figlio di VGGFace e a che serve} che garantisca le migliori performance possibili sul test set associato a questo dataset, rispettando i seguenti requisiti:

\begin{itemize}
	\item non richieda, per quanto possibile, risorse computazionali eccessive, dal momento che per l'addestramento sarà utilizzata la piattaforma Google Colab\footnote{\url{https://colab.research.google.com/}}, che offre potenza computazionale limitata e permette l'accesso alle sue risorse per un massimo di 12 ore consecutive, per poi sottoporre l'utente ad un ban temporaneo; \todo{Attenzione ai bullet point all'italiana}
	\item sia robusta rispetto a corruzioni che possono verificarsi su immagini acquisite con un'intenzione simile a quella con la quale sono state acquisite le immagini presenti nel dataset.
Essendo le immagini nel dataset immagini di celebrità recuperate su Google Immagini \cite{vggface2dataset}, si assumerà come intenzione quella di un fotografo che vuole realizzare uno scatto di un soggetto più o meno collaborativo.
	\item (possibilmente) sia implementata con il framework Keras.\todo{aggiungere motivazioni}
\end{itemize}

Il codice per l’addestramento e il test della rete realizzata nell’ambito di questo progetto è consultabile sulla nostra repository GitHub\footnote{\url{https://github.com/MicheleOliva/Contest-Vision-2020}}.

\section{Descrizione della soluzione}
Tale descrizione deve contemplare almeno gli elementi riportati nelle seguenti
sottosezioni.

\subsection{Convolutional neural network}
\emph{Descrivere in questa sezione l’architettura della convolutional neural network
sviluppata nell’ambito del progetto. Se si utilizza un’architettura nota, si riportino gli elementi fondamentali della rete e le eventuali modifiche effettuate ma, soprattutto, si motivi la scelta.
Definire in ogni caso se la rete è stata progettata come regressore (output:
numero reale da 0 a 100 che rappresenta l’età, da approssimare all’intero più
vicino) o come classificatore (output: una delle 101 classi da 0 a 100) e fornire dettagli e motivazioni sulla funzione di costo scelta.}

L'architettura scelta è MobileNetV3.\todo{citazione del MIVIA?} Le reti MobileNet sono conosciute per essere progettate pensando all'efficienza computazionale e alla latenza delle predizioni, senza però sacrificare le prestazioni. MobileNetV3 migliora MobileNetV2 sia in termini di accuracy che di latenza, diminuendo il numero di pesi ed il numero di operazioni totali. Questo ci ha permesso di allenare la rete e di testarla in tempi ragionevoli rapportati al tempo disponibile, e quindi ha permesso una scelta migliore dei parametri.
La rete è stata progettata come un regressore. Abbiamo sostituito il layer di classificazione con un layer di dimensione unitaria con attivazione lineare (ReLU). Questo ci ha permesso di astrarci dalla rappresentazione dell'età, che non è per forza definita nell'intervallo \([0, 100]\), e l'utilizzo della ben nota in letteratura Mean Squared Error (MSE) loss function, che tiene anche conto della differenza tra la stima dell'età e la \emph{ground truth}.

\subsection{Procedura di addestramento}
\subsubsection{Dataset}
\label{subsubsec:dataset}

Il dataset utilizzato per allenare la nostra rete a compiere il task assegnato è il dataset chiamato \emph{VGG-Face2 Mivia Age}.

Quest'ultimo dataset contiene $3.31$ milioni di immagini di 9131 soggetti, con una media di $362.6$ immagini per ogni soggetto. Le immagini sono scaricate da Google Image Search e si diversificano molto per posa, età, illuminazione, etnia e professione del soggetto catturato. Ad ogni immagine è associata l'età del soggetto, tale valore è il risultato di un ensemble di $14$ modelli CNN.\\
Il dataset ci è stato fornito già diviso in training set, che include $8631$ identità, e test set, che include le rimanenti 500.

\todo{Non mi piace la seguente frase, aiuto} Le risorse che \textit{Google Colab} mette a disposizione sono risultate insufficienti per allenare la rete sull'intero training set. È risultato necessario ridurre la grandezza di quest'ultimo, a tale scopo è stata svolta un analisi del training set e sono stati scelte le identità su cui svolgere l'addestramento della rete. Tale analisi è stata svolta anche con il supporto delle annotazioni per VGGFace2, reperibili \href{https://github.com/MiviaLab/GenderRecognitionFramework/releases/tag/0}{qui}.

Con lo scopo di svolgere l'allenamento su un training set più bilanciato possibile, abbiamo svolto una prima analisi sul genere delle identità presenti.

\begin{figure}[H]

\begin{subfigure}{0.5\textwidth}
\def\svgscale{0.5}
\input{./Images/gender_ids.pdf_tex}
\caption{Identità divise per genere}
\label{sfig:Ids per gender}
\end{subfigure}
\begin{subfigure}{0.5\textwidth}
\def\svgscale{0.5}
\input{./Images/gender_images.pdf_tex}
\caption{Immagini divise per genere}
\label{sfig:Images per gender}
\end{subfigure}
\caption{Divisione per genere di identità e immagini}
\label{fig:gender_division}
\end{figure}

Come possiamo vedere in \ref{sfig:Ids per gender} e in \ref{sfig:Images per gender}, il training set risulta sbilanciato per quanto riguarda il genere dei soggetti.

L'analisi si è successivamente concentrata sulla \emph{media e deviazione standard} dell'età di ogni soggetto presente nel training set.

\begin{figure}[H]

\begin{subfigure}{0.5\textwidth}
\def\svgscale{0.42}
\input{./Images/n_ids_by_age_and_gender.pdf_tex}
\caption{Identità divise per genere ed età media}
\label{sfig:Ids per gender and mean age}
\end{subfigure}
\begin{subfigure}{0.5\textwidth}
\def\svgscale{0.42}
\input{./Images/n_images_by_age_and_gender.pdf_tex}
\caption{Immagini divise per genere ed età media}
\label{sfig:Images per gender and mean age}
\end{subfigure}
\caption{Divisione per genere di identità e immagini}
\label{fig:gender_age_division}
\end{figure}

Come si evince dai grafici in figura~\ref{sfig:Ids per gender and mean age} e in figura~\ref{sfig:Images per gender and mean age}, ci sono delle fasce d'età sovrarappresentate rispetto alle altre, in particolare le fasce d'età $[25,34]$ e $[35,44]$.

Con lo scopo di avere un training set quanto più bilanciato possibile, abbiamo scelto di escludere dal set gli uomini nelle fasce: $[25,34]$ e $[35,44]$; e le donne nella fascia $[25,34]$; la cui deviazione standard dell'età risulta inferiore a $5.5$. Il risultato di tale operazione è visibile in figura~\ref{fig:Ids per gender and mean age after the drop}.

\begin{figure}[H]
\centering
\def\svgscale{0.7}
\input{./Images/n_ids_by_age_and_gender_after_drop.pdf_tex}
\caption{Identità divise per genere ed età media dopo il taglio}
\label{fig:Ids per gender and mean age after the drop}
\end{figure}

Successivamente a tale operazione, al training set sono state sottratte ulteriori $500$ identità, scelte casualmente, che hanno formato il \emph{validation set}.

\subsubsection{Face detection}
\label{subsubsec:face_detection}

\emph{Descrivere il metodo utilizzato per effettuare il rilevamento del volto. Se si utilizza un approccio noto o i volti già estratti con framework esistenti, si specifichi questa informazione.}

Non è stato effettuato alcun rilevamento del volto. Nelle annotazioni del dataset VGGFAce2, già citate nella sezione~\ref{subsubsec:dataset}, sono riportate sia per il training set che per il test set, informazioni inerenti alle bounding boxes relative ai volti dei soggetti catturati. Tali informazioni definiscono quindi la \textbf{region of interest (\textbf{roi})} di ogni immagine.

\subsubsection{Face pre-processing} 

\emph{Descrivere e motivare tutte le tecniche di pre-processing applicate sulle immagini del volto.}

Si distinguono due fasi del pre-processing delle immagini, la prima fase viene effettuata a monte della data augmentation mentre la seconda a valle di quest'ultima.
Durante la prima fase viene ritagliato il volto del soggetto. Il ritaglio non viene effettuato però sulla roi (sezione~\ref{subsubsec:face_detection}), infatti la region of interest viene allargata di un fattore $0.3$, ovviamente non andando oltre i limiti dettati dalla risoluzione dell'immagine, tale operazione permette di ritagliare oltre al volto del soggetto l'intera testa.
Nella seconda fase ogni immagine viene prima ridimensionata alla risoluzione \todo{Attenzione alla risoluzione} $96 \times 96$ mantenendo le proporzioni, nel caso in cui l'immagine non fosse quadrata viene \todo{in che senso riempita?} riempita con delle bande nere. Successivamente il valore medio di ogni canale è sottratto da ogni pixel, ed infine viene convertita da BGR ad RGB.

\subsubsection{Data augmentation}
Descrivere e motivare tutte le policy di augmentation
implementate per estendere il dataset o per aumentarne la rappresentativit`a.

\subsection{Training from scratch o fine tuning}
Specificare se la rete viene addestrata
con inizializzazione random o partendo da pesi pre-addestrati, motivando la
scelta e fornendo dettagli sulla strategia di inizializzazione.

\subsubsection{Procedura di training}
Dettagliare e motivare almeno le seguenti scelte: numero di epoche di addestramento, tipo di ottimizzatore, learning rate scheduling (tecnica di riduzione, learning rate iniziale, fattore di riduzione). Fornire dettagli su eventuali elementi aggiuntivi: batch normalization, weight decay, early stopping etc. Per ognuna delle scelte, riportare i valori esatti dei parametri utilizzati, per rendere l’esperimento riproducibile. Motivare la scelta di tali valori.

\section{Risultati sperimentali}
\emph{Descrivere gli esperimenti effettuati e, per ognuno di essi, riportare i risultati sul training e sul validation set in forma tabellare. Analizzare e commentare i risultati nel dettaglio, tirando fuori delle conclusioni motivate dai risultati sperimentali. Eventuali esperimenti aggiuntivi possono essere riportati in questa sezione.}

\section{Conclusioni}
\begin{comment}
Riportare nella relazione le conclusioni del lavoro svolto.

- MobileNet funziona bene dal punto di vista del MAE nonostante sia stata ottimizzata per una specifica applicazione (utilizzo mobile/embedded, low power etc.)
- Tutte le reti arrivano ad un certo punto più o meno simile e si fermano: potremmo aver trovato un limite, ma per saperlo bisognerebbe fare più prove, cambiando tecnica di warmup, riduzione del LR, loss function etc.
- Il classificatore merita attenzioni
- Sarebbe interessante testare la rete su di un dataset annotato con le età esatte ma da umani anziché da una rete neurale
- Sarebbe interessante testarla in the wild, per valutare l'efficacia della data augmentation 
- Si possono valutare le prestazioini insieme ad un face detector
\end{comment}


In questo progetto abbiamo evidenziato come MobileNet, se allenata su di un dataset sufficientemente vasto, possa avere ottime performance di generalizzazione nonostante si tratti di una rete ottimizzata per l'ambito embedded/mobile.

Per quanto riguarda le reti da noi analizzate, un elemento interessante è dato dal fatto che tutte raggiungono prestazioni sul validation set più o meno simili e dopo smettono di migliorare. Questo potrebbe significare che le prestazioni da noi rilevate rappresentano una sorta di \emph{lower bound} per l'architettura; tuttavia, per rendere effettiva tale affermazione, andrebbero effettuate molte altre prove, utilizzando tutto il dataset e variando la procedura di addestramento per quanto riguarda il numero di step di warmup, il learning rate di regime e la politica di scheduling dello stesso, la loss function, i livelli più alti della rete.

Come possibile sviluppo futuro, proponiamo un'analisi più approfondita del modello basato su classificatore, sia in termini di procedure di addestramento che di misurazione delle performance, visto che quest'ultime sembrano molto promettenti.

Considerato che il dataset VMAGE è stato annotato tramite l'utilizzo di una rete neurale, che, per quanto performante, è comunque affetta da errore, risulta complicato per noi dare una stima robusta delle performance delle nostre reti in un'applicazione reale. Proponiamo, quindi, l'analisi delle prestazioni sia su dataset diversi, ma annotati comunque con sufficiente precisione, che in applicazioni ``in the wild''. Per quest'ultima, è necessario un approfondimento della fase di data augmentation, che risulta importante per l'invarianza rispetto alle corruzioni delle immagini tipiche di questa situazione \cite{miviagender}.

Infine, proponiamo la progettazione di un sistema che comprenda anche un face detector, la cui efficacia è alla base del funzionamento delle nostre reti, e la valutazione delle performance dell'intera pipeline, tenendo conto dell'intenzione dietro la progettazione di MobileNet.




\printbibliography
\end{document}
