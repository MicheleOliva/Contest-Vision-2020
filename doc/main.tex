\documentclass[a4paper, 11pt]{article}
\usepackage{layaureo}
%Pacchetti per l'italiano
\usepackage{lmodern}
\usepackage[T1]{fontenc}
\usepackage[utf8]{inputenc}
\usepackage[italian]{babel}
\usepackage{csquotes}
% Pacchetto per eliminare le giunture
\usepackage{microtype}
% pacchetto teoremi ecc.
\usepackage{amsthm}
% Pacchetto matematica
\usepackage{amsmath}
\usepackage{amssymb}
% Pacchetto per gestire meglio i float (H e box)
\usepackage{float}
% pacchetto per aggiungere figure grafiche
\usepackage{graphicx}
% Pacchetto per aumentare il limite di float
\usepackage{morefloats}
% pacchetto per modificare le caption
\usepackage{caption}
% Pacchetto gestione dei hyperlinks
\usepackage{hyperref}
%liste in più
\usepackage{scrextend}
%liste inline
\usepackage[inline]{enumitem}
%autor
\usepackage{authblk}
% TO DO
\usepackage{todonotes}

\graphicspath{{Images/}}
\usepackage{wrapfig}
\usepackage{subcaption}
\usepackage{calc}
\usepackage{gensymb} % Aggiunge il simbolo dei gradi

% Bibliografia
\usepackage[
backend=biber,
style=alphabetic,
sorting=ynt
]{biblatex}

%authblk in Italiano
\renewcommand\Authand{ e }
\renewcommand\Authands{ e }

%colora i link
\hypersetup{
    colorlinks=true,
    linkcolor=black,
    urlcolor=blue,
}

%Bibliografia
\addbibresource{biblio.bib}

% Titolo
\title{Contest di Visione Artificiale: Gruppo 2}
\author{Luca Boccia}
\author{Francesco Chiarello}
\author{Michele Oliva}
\affil{\texttt{\{\href{mailto:l.boccia12@studenti.unisa.it}{l.boccia12}, \href{mailto:f.chiarello1@studenti.unisa.it}{f.chiarello1}, \href{mailto:m.oliva26@studenti.unisa.it}{m.oliva26}\}@studenti.unisa.it}}
\affil{Università degli Studi di Salerno}
\date{Gennaio 2021}


\begin{document}

\maketitle
\tableofcontents
\listoffigures

\section{Introduzione}
In questo progetto affrontiamo il problema della stima dell'età di un essere umano a partire da immagini del volto attraverso l'utilizzo di reti neurali profonde. 
In particolare, lo scopo che ci prefiggiamo è quello di addestrare una rete neurale profonda sul dataset VGGFace2 \cite{vggface2dataset} che garantisca le migliori performance possibili sul test set associato a questo dataset, rispettando i seguenti desiderata:

\begin{itemize}
	\item non richieda, per quanto possibile, risorse computazionali eccessive, dal momento che per l'addestramento sarà utilizzata la piattaforma \href{https://colab.research.google.com/notebooks/intro.ipynb}{Google Colab}, che offre potenza computazionale limitata e permette l'accesso alle sue risorse per un massimo di 12 ore consecutive, per poi sottoporre l'utente ad un ban temporaneo;
	\item sia robusta rispetto a corruzioni che possono verificarsi su immagini acquisite con un'intenzione simile a quella con la quale sono state acquisite le immagini presenti nel dataset.
Essendo le immagini nel dataset immagini di celebrità recuperate su Google Immagini \cite{vggface2dataset}, si assumerà come intenzione quella di un fotografo che vuole realizzare uno scatto di un soggetto più o meno collaborativo.
	\item (possibilmente) sia implementata con il framework Keras.
\end{itemize}

Il codice per l’addestramento e il test della rete realizzata nell’ambito di questo progetto è consultabile sulla nostra \href{https://github.com/MicheleOliva/Contest-Vision-2020}{repository GitHub}. 


\section{Descrizione della soluzione}
Tale descrizione deve contemplare almeno gli elementi riportati nelle seguenti
sottosezioni.

\subsection{Convolutional neural network}
Descrivere in questa sezione l’architettura della convolutional neural network
sviluppata nell’ambito del progetto. Se si utilizza un’architettura nota, si riportino gli elementi fondamentali della rete e le eventuali modifiche effettuate ma, soprattutto, si motivi la scelta.

Definire in ogni caso se la rete `e stata progettata come regressore (output:
numero reale da 0 a 100 che rappresenta l’età, da approssimare all’intero più
vicino) o come classificatore (output: una delle 101 classi da 0 a 100) e fornire dettagli e motivazioni sulla funzione di costo scelta.

\subsection{Procedura di addestramento}
\subsubsection{Dataset}

Il dataset utilizzato per allenare la nostra rete a compiere il task assegnato è il dataset chiamato \emph{VGG-Face2 Mivia Age}.

Quest'ultimo dataset contiene $3.31$ milioni di immagini di 9131 soggetti, con una media di $362.6$ immagini per ogni soggetto. Le immagini sono scaricate da Google Image Search e si diversificano molto per posa, età, illuminazione, etnia e professione del soggetto catturato. Ad ogni immagine è associata l'età del soggetto, tale valore è il risultato di un ensemble di $14$ modelli CNN.\\
Il dataset ci è stato fornito già diviso in training set, che include $8631$ identità, e test set, che include le rimanenti 500.

\todo{Non mi piace la seguente frase, aiuto} Le risorse che \textit{Google Colab} mette a disposizione 

Dettagliare e motivare le scelte relative alla preparazione del dataset, ovvero alla composizione del training e del validation set. Se si sceglie di utilizzare un sottoinsieme del training set per ridurre i tempi di addestramento (soluzione consigliata), si motivi la scelta del numero e del tipo di campioni utilizzati per l’addestramento. Descrivere e motivare il protocollo sperimentale utilizzato per valutare le performance sul validation set (es. cross-validation).

\subsubsection{Face detection} 
Descrivere il metodo utilizzato per effettuare il rilevamento del volto. Se si utilizza un approccio noto o i volti già estratti con framework esistenti, si specifichi questa informazione.

\subsubsection{Face pre-processing} 
Descrivere e motivare tutte le tecniche di pre-processing applicate sulle immagini del volto.

\subsubsection{Data augmentation}
Descrivere e motivare tutte le policy di augmentation
implementate per estendere il dataset o per aumentarne la rappresentativit`a.

\subsection{Training from scratch o fine tuning}
Specificare se la rete viene addestrata
con inizializzazione random o partendo da pesi pre-addestrati, motivando la
scelta e fornendo dettagli sulla strategia di inizializzazione.

\subsubsection{Procedura di training}
Dettagliare e motivare almeno le seguenti scelte: numero di epoche di addestramento, tipo di ottimizzatore, learning rate scheduling (tecnica di riduzione, learning rate iniziale, fattore di riduzione). Fornire dettagli su eventuali elementi aggiuntivi: batch normalization, weight decay, early stopping etc. Per ognuna delle scelte, riportare i valori esatti dei parametri utilizzati, per rendere l’esperimento riproducibile. Motivare la scelta di tali valori.

\section{Risultati sperimentali}
\begin{comment}
    Descrivere gli esperimenti effettuati e, per ognuno di essi, riportare i risultati sul training e sul validation set in forma tabellare. Analizzare e commentare i risultati nel dettaglio, tirando fuori delle conclusioni motivate dai risultati sperimentali. Eventuali esperimenti aggiuntivi possono essere riportati in questa sezione.
\end{comment}

\subsection{Descrizione esperimenti}

\begin{comment}
    3 esperimenti:
    - regressore 224x244
    - classificatore 224x224
    - regressore 96x96
    Come cosa aggiuntiva, epoche non full che fanno comunque paura
\end{comment}

Abbiamo considerato nei nostri esperimenti tre tipi di reti, tutte basate su MobileNetV3 Large:
\begin{itemize}
    \item regressore con dimensione di input $96 \times 96$;
    \item regressore con dimensione di input $224 \times 224$;
    \item classificatore con dimensione di input $224 \times 224$.
\end{itemize}

Siccome eseguire un'epoca di addestramento, con successiva validazione su tutto il nostro validation set, richiede un tempo considerevole (almeno 2 ore e 30 minuti, anche impostando come input size $96 \times 96$), al fine di avere dei primi feedback riguardo l'appropriatezza della nostra procedura di training in tempi ragionevoli, abbiamo condotto inizialmente degli addestramenti che definiamo \emph{light}. Questi sono caratterizzati da epoche che non esplorano tutto il training set, ma un suo sottoinsieme (causale per ogni epoca) tale da contenere un numero di immagini multiplo del numero di identità, in modo che l'insieme dei dati sottoposti alla rete sia comunque bilanciato da questo punto di vista.
Per questioni di tempo, abbiamo seguito questo approccio soltanto per allenare i due diversi regressori. I risultati hanno evidenziato come entrambe le reti promettessero buone performance, e, pertanto, abbiamo proseguito con un loro addestramento "classico" (in seguito chiamato \emph{full}), vale a dire utilizzando training e validation set completi.

Come ultimo esperimento, abbiamo addestrato un classificatore con 101 classi (da 0 a 100 anni) ed input size $224 \times 224$. Come già detto, su quest'ultimo non abbiamo eseguito molte prove per mancanza di tempo.

\subsection{Risultati e commenti}

I risultati ottenuti sul nostro validation set sono riportati nella tabella seguente. Per completezza, riportiamo anche i risultati dei modelli addestrati con procedura \emph{light}. 

\begin{table}[ht]
    \centering
    \begin{tabular}{ |c|c|c|c|c| } 
        \hline
        \textbf{Tipologia} & \textbf{Training} & \textbf{Input Size} & \textbf{Epoche} & \textbf{val.\@ MAE} \\
        \hline
        Regressore & \emph{light} & $96 \times 96$  & $43$ & $2.47$ \\
        \hline
        Regressore & \emph{light} & $224 \times 224$ & $77$ & $1.84$ \\
        \hline
        Regressore & \emph{full} & $96 \times 96$ & $20$ & $2.44$ \\
        \hline
        Regressore & \emph{full} & $224 \times 224$ & $14$ & $1.75$ \\
        \hline   
        Classificatore & \emph{full} & $224 \times 224$ & $8$ & $1.64$ \\
        \hline
    \end{tabular}
    \caption{Risultati degli esperimenti effettuati. Il MAE indicato in tabella fa riferimento al validation set completo anche per i modelli addestrati con procedura \emph{light}.}
\end{table}

In primo luogo notiamo come le prestazioni dei modelli addestrati su epoche \emph{light} siano molto simili alle loro controparti addestrate sull'intero training set. Questo mostra quanto creare dei batch bilanciati dal punto di vista delle identità ed eseguire ad ogni epoca un mescolamento dei dati sia importante affinché la rete impari feature robuste e non si faccia trascinare da pattern eventualmente presenti nella sequenza di dati. Vale la pena notare, comunque, che il numero di epoche per i modelli sottoposti a training \emph{light} è di gran lunga maggiore rispetto ai modelli addestrati con tecnica \emph{full}, di conseguenza, statisticamente, i modelli \emph{light} "visualizzano" un numero di immagini paragonabile ai modelli \emph{full}, quindi possiamo concludere che la convergenza avviene in tempi simili.

Il secondo elemento di riflessione è dato dal fatto che il regressore con input size $96 \times 96$ presenta un MAE sul validation set inferiore ad un anno rispetto a quello con input size $224 \times 224$. Questo è frutto sicuramente delle buone capacità di generalizzazione ottenute nonostante la minore input size, ma anche del fatto che molte immagini nel dataset hanno dimensioni davvero ridotte, per cui sono relativamente pochi i campioni sui quali una maggiore dimensione di input riesce a fare la differenza.

\todo{Il classificatore è forte, Tempi di esecuzione}

\subsection{Riguardo il classificatore}

Il classificatore che abbiamo addestrato ha la stessa struttura dei regressori eccetto che per l'ultimo livello, il quale presenta un output di dimensione 101, corrispondente alle età da 0 a 100. 

L'approccio che abbiamo utilizzato per la codifica della groundtruth e degli output, e per il calcolo della funzione di costo, è quello denominato \emph{ordinal regression} \cite{ordinalregression}. Esso prevede che vi sia una relazione d'ordine tra le classi, e che queste siano rappresentate come vettori di interi a valori in $\left\{0,1\right\}^n$, i quali assumono valore 1 fino alla classe di appartenenza di un campione, e zero dalla classe successiva in poi. Quindi, ad esempio, su 10 classi, la classe 3 viene rappresentata come:
\begin{displaymath}
    \begin{bmatrix}
    1 & 1 & 1 & 1 & 0 & 0 & 0 & 0 & 0 & 0
    \end{bmatrix}.
\end{displaymath}
Per quanto riguarda l'output del modello, ogni neurone dell'ultimo livello utilizza la funzione d'attivazione sigmoide, così come previsto in \cite{ordinalregression}, quindi la rete produce un vettore di probabilità. La classe predetta (l'età, nel nostro caso) sarà pari all'indice del primo elemento dell'output con valore minore di 0.5 diminuito di 1. 

In \cite{ordinalregression} si suggerisce di utilizzare come funzioni di costo l'errore quadratico tra previsione e groundtruth oppure la binary-crossentropy, e si specifica che entrambe producono risultati molto simili nella maggior parte dei casi. Noi abbiamo optato per la binary-crossentropy. 

Per quanto riguarda la procedura di training, è la stessa utilizzata per i regressori, valori dei vari parametri inclusi, l'unico elemento che differisce è il fattore di riduzione del learning rate, che qui è pari a $0.1$, in quanto dalle prime prove sembrava garantire una convergenza leggermente più veloce.


\section{Conclusioni}
\begin{comment}
Riportare nella relazione le conclusioni del lavoro svolto.

- MobileNet funziona bene dal punto di vista del MAE nonostante sia stata ottimizzata per una specifica applicazione (utilizzo mobile/embedded, low power etc.)
- Tutte le reti arrivano ad un certo punto più o meno simile e si fermano: potremmo aver trovato un limite, ma per saperlo bisognerebbe fare più prove, cambiando tecnica di warmup, riduzione del LR, loss function etc.
- Il classificatore merita attenzioni
- Sarebbe interessante testare la rete su di un dataset annotato con le età esatte ma da umani anziché da una rete neurale
- Sarebbe interessante testarla in the wild, per valutare l'efficacia della data augmentation 
- Si possono valutare le prestazioini insieme ad un face detector
\end{comment}


In questo progetto abbiamo evidenziato come MobileNet, se allenata su di un dataset sufficientemente vasto, possa avere ottime performance di generalizzazione nonostante si tratti di una rete ottimizzata per l'ambito embedded/mobile.

Per quanto riguarda le reti da noi analizzate, un elemento interessante è dato dal fatto che tutte raggiungono prestazioni sul validation set più o meno simili e dopo smettono di migliorare. Questo potrebbe significare che le prestazioni da noi rilevate rappresentano una sorta di \emph{lower bound} per l'architettura; tuttavia, per rendere effettiva tale affermazione, andrebbero effettuate molte altre prove, utilizzando tutto il dataset e variando la procedura di addestramento per quanto riguarda il numero di step di warmup, il learning rate di regime e la politica di scheduling dello stesso, la loss function, i livelli più alti della rete.

Come possibile sviluppo futuro, proponiamo un'analisi più approfondita del modello basato su classificatore, sia in termini di procedure di addestramento che di misurazione delle performance, visto che quest'ultime sembrano molto promettenti.

Considerato che il dataset VMAGE è stato annotato tramite l'utilizzo di una rete neurale, che, per quanto performante, è comunque affetta da errore, risulta complicato per noi dare una stima robusta delle performance delle nostre reti in un'applicazione reale. Proponiamo, quindi, l'analisi delle prestazioni sia su dataset diversi, ma annotati comunque con sufficiente precisione, che in applicazioni ``in the wild''. Per quest'ultima, è necessario un approfondimento della fase di data augmentation, che risulta importante per l'invarianza rispetto alle corruzioni delle immagini tipiche di questa situazione \cite{miviagender}.

Infine, proponiamo la progettazione di un sistema che comprenda anche un face detector, la cui efficacia è alla base del funzionamento delle nostre reti, e la valutazione delle performance dell'intera pipeline, tenendo conto dell'intenzione dietro la progettazione di MobileNet.




\printbibliography
\end{document}
