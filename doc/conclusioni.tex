\begin{comment}
Riportare nella relazione le conclusioni del lavoro svolto.

- MobileNet funziona bene dal punto di vista del MAE nonostante sia stata ottimizzata per una specifica applicazione (utilizzo mobile/embedded, low power etc.)
- Tutte le reti arrivano ad un certo punto più o meno simile e si fermano: potremmo aver trovato un limite, ma per saperlo bisognerebbe fare più prove, cambiando tecnica di warmup, riduzione del LR, loss function etc.
- Il classificatore merita attenzioni
- Sarebbe interessante testare la rete su di un dataset annotato con le età esatte ma da umani anziché da una rete neurale
- Sarebbe interessante testarla in the wild, per valutare l'efficacia della data augmentation 
- Si possono valutare le prestazioini insieme ad un face detector
\end{comment}


In questo progetto abbiamo evidenziato come MobileNet, se allenata su di un dataset sufficientemente vasto, possa avere ottime performance di generalizzazione nonostante si tratti di una rete ottimizzata per l'ambito embedded/mobile.

Per quanto riguarda le reti da noi analizzate, un elemento interessante è dato dal fatto che tutte raggiungono prestazioni sul validation set più o meno simili e dopo smettono di migliorare. Questo potrebbe significare che le prestazioni da noi rilevate rappresentano una sorta di \emph{lower bound} per l'architettura; tuttavia, per rendere effettiva tale affermazione, andrebbero effettuate molte altre prove, utilizzando tutto il dataset e variando la procedura di addestramento per quanto riguarda il numero di step di warmup, il learning rate di regime e la politica di scheduling dello stesso, la loss function, i livelli più alti della rete.

Come possibile sviluppo futuro, proponiamo un'analisi più approfondita del modello basato su classificatore, sia in termini di procedure di addestramento che di misurazione delle performance, visto che quest'ultime sembrano molto promettenti.

Considerato che il dataset VMAGE è stato annotato tramite l'utilizzo di una rete neurale, che, per quanto performante, è comunque affetta da errore, risulta complicato per noi dare una stima robusta delle performance delle nostre reti in un'applicazione reale. Proponiamo, quindi, l'analisi delle prestazioni sia su dataset diversi, ma annotati comunque con sufficiente precisione, che in applicazioni ``in the wild''. Per quest'ultima, è necessario un approfondimento della fase di data augmentation, che risulta importante per l'invarianza rispetto alle corruzioni delle immagini tipiche di questa situazione \cite{miviagender}.

Infine, proponiamo la progettazione di un sistema che comprenda anche un face detector, la cui efficacia è alla base del funzionamento delle nostre reti, e la valutazione delle performance dell'intera pipeline, tenendo conto dell'intenzione dietro la progettazione di MobileNet.

